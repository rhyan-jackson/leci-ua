%! Author = rhyan
%! Date = 17/10/23

% Preamble
\documentclass{report}
\usepackage[utf8]{inputenc}
%\usepackage{showlabels}
\usepackage{textcomp}
\usepackage{graphicx}
% Capa
\title{\textbf{Entendendo \LaTeX}}
\author{Rhyan Jackson Marinho Matos}
\date{\today}
% Documento
\begin{document}
    \maketitle
    \tableofcontents
    \part[Listas]{Listas ordenadas e não ordenadas}\label{part:listas}
    
    
    \chapter{Não ordenadas}\label{chap:nao-ordenadas}
            \section{Itemize}\label{sec:itemize}
                Aqui está a lista:
                \begin{itemize}\label{lista-itemize-exemplo}
                    \item primeiro item.
                    \item segundo item.
                    \item terceiro item.
                    \item quarto item.
                \end{itemize}
                Mas, pode-se trocar a forma de visualização:
                \begin{itemize}\label{lista-itemize-outravisu-exemplo}
                    \item[+] item 1.
                    \item[o] item 2.
                    \item[*] item 3.
                \end{itemize}
            \section{Descriptions}\label{sec:descriptions}
                Aqui está o outro tipo de lista:
                \begin{description}\label{lista-desc-exemplo}
                    \item[UA] - TESTE
                    \item[UB] - TESTE
                    \item[UC] - TESTE
                \end{description}
                Aqui retorna a indentação.
    
        \chapter{Ordenadas}\label{chap:ordenadas}
            \section{Enumerate}\label{sec:enumerate}
                Enumerate padrão:
                \begin{enumerate}\label{lista-enumerate-exemplo}
                    \item Item 1
                    \item Item 2
                    \item Item 3
                \end{enumerate}
                Enumerate com números romanos:
                \renewcommand{\theenumi}{\Roman{enumi}}
                \begin{enumerate}\label{lista-enumerate-exemplo-interna}
                    \item Item 1
                    \item Item 2
                    \item Item 3
                \end{enumerate}
    
        \chapter{Mix}\label{chap:mix}
            \renewcommand{\theenumi}{\Roman{enumi}}
            \begin{enumerate}\label{lista-enumerate-mix}
                \item Este é um item, com subitems
                    \renewcommand{\theenumi}{\alpha{enumi}}
                    \begin{enumerate}\label{lista-enumerate-mix-interna}
                        \item Este é um item
                        \item Este é outro item
                        \item etc.
                    \end{enumerate}
                \renewcommand{\theenumi}{\Roman{enumi}}
                \item Este é outro item
                \item etc.
            \end{enumerate}
            Aqui retorna a indentação
    
    \part[Imagens e Tabelas]{Utilização de Imagens e Tabelas}\label{part:imagens-e-tabelas}
    
        \chapter{Imagens}\label{chap:imagens}
            Termos para o begin figure:
                \begin{description}\label{lista-termos-figure}
                    \item[h] - a figura fica aonde o comando foi dado.
                    \item[t] - a figura fica no topo da página.
                    \item[b] - a figura fica no fundo da página.
                    \item[p] - a figura fica numa página isolada.
                \end{description}
            \begin{figure}[h]\label{figura-imagens}
            \centerline{\fbox{Conteúdo da figura.}}
            \caption{Legenda da figura}
            \end{figure}
        \chapter{Tabelas}\label{exemplo-tabela}
            \begin{table}[h]\label{tabela-exemplo}
                \caption{Exemplo de uma tabela}\label{cap-exemplo-tabela}
                    \centerline{Conteúdo da tabela}
            \end{table}

    Testando a referência:~\ref{figura-imagens}
    
    \part[Tabular e Matrizes]{Utilização de Tabulares e Matrizes}\label{part:tabular-matrizes}
        
        \chapter{Tabular}\label{chap:tabular}
            \begin{tabular}{|l||c|r|} % Alinhamento: Esquerda, Centro, Direita
                
                \hline
                    & Temperatura & Umidade \\
            Cidade  & (\textordmasculine C) & (perc.) \\ \hline\hline
            Aveiro  & 10 & 90 \\ \hline
            Lisboa  & 9 & 89 \\ \hline
            \end{tabular}
    
    \part[Imagens]{Inserção de imagens no \LaTeX}\label{part:imagens-graphics}
        
        \chapter{Include Graphics}\label{chap:include-graphics}
            \begin{figure}[h]\label{imagem-exemplo}
                \center
                \includegraphics[height=150pt]{test}
                \caption{Legenda da figura}
            \end{figure}
            As imagens têm várias opções possíveis ao serem colocadas no documento, são elas:
            \begin{description}\label{opcoes-imagem-lista}
                \item[height] - altura da imagem.
                \item[width] - largura da imagem.
                \item[scale] - fator de escala.
                \item[angle] - ângulo de rotação.
            \end{description}
            
            Um outro exemplo de uso:
            \begin{figure}[h]
                \center% Centra as imagens
                    a)\includegraphics[height=10cm]{test}
                    b)\includegraphics[height=3cm]{test}
                    c)\includegraphics[width=20mm]{test}
                    d)\includegraphics[scale=.5,angle=90]{test}
                    e)\includegraphics[height=5mm,width=3cm]{test}
                \caption{Logotipo da Universidade de Aveiro:
                a) na dimensão real,
                    b) com 3cm de altura,
                    c) com 20mm de largura,
                    d) com altura e largura reduzidas a $1/2$ e simultaneamente rodado 90ºe
                    e) com uma modificação anamórfica da altura e da largura.}
                \label{fig:ualogo.2}
            \end{figure}
    
    
    
    
    
    \end{document}